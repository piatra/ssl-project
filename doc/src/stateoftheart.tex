\section{State of the Art}

Research in the area of demographics extraction from web resources is centered around two main approaches. The first approach deals with the content the users are posting and reading. The downside of this approach is that the majority of users don't create any content and are just consumers. The second approach aims to fix this problem by analyzing the users' browsing behaviour by treating the words and hyperlinks and other web page components as hidden attributes.

A solution that belongs to the first category of algorithms proposes the demographic analysis on social media posts\cite{Marquardt}. The features they use are split in context-based features(MRC features, LIWC features, sentiment) and stylistic features(readability, HTML Tags, spelling and grammatical errors, emoticons). Using these features they build models based on Support Vector Machines(SVM) algorithms. Their method didn't reach the desired results, one of their classifiers scoring under the base line algorithm.

SVM solutions are very popular for solving age and gender detection problem based on text. Solutions \cite{Hu} and \cite{Kabbur} also use this kind of algorithms to address the problem. The age range is split in a discrete of categoires: eg. $<18$(teenager), $18-24$(youngster), $25-34$(young), $35-49$(mid-age), $>49$(elder). Features are again split in two categories content-based and category-based and a SVM classifier is trained using the data.

D. Nguyen, N. Smith and C. Rose\cite{Nguyen} propose a solution based on corpus analysis.  Their dataset contained a Blog corpus obtained from crawling blogs and annotating them with gender and age, the Fisher telephone corpus containing transcripts of telephone conversations and posts from a Breast Cancer forum. The features extracted from this data was split in two categories: textual features (unigrams, POS unigrams and bigrams, LIWC for word counting) and gender. Because data comes from different sources they used a Joint model and each feature was mapped to four categories from which one is global and three are corpus specific. The prediction is done using linear regression and obtained a correlation up to 0.74 which is a good result regarding the fact that linear regression is a new solution in the field.

A good example of the second category approaches is the one proposed by D. Phuong, T. Phuong\cite{Phuong} for gender detection. For a set of features, they are splitting the web sites from a browsing history in a set of predefined categories. Another set of features is obtained by tagging a website with a topic based on its content which is done automatically by a classifier. To complete the set of features time and visiting sequence are added to the set of features. Using these features a SVM model is built. The results are very promising with an average F1 score greater than 0.7
 